% Generated by Sphinx.
\def\sphinxdocclass{report}
\documentclass[letterpaper,10pt,english]{sphinxmanual}
\usepackage[utf8]{inputenc}
\DeclareUnicodeCharacter{00A0}{\nobreakspace}
\usepackage[T1]{fontenc}
\usepackage{babel}
\usepackage{times}
\usepackage[Bjarne]{fncychap}
\usepackage{longtable}
\usepackage{sphinx}
\usepackage{multirow}


\title{CABS Documentation}
\date{July 25, 2012}
\release{2012}
\author{Andrzej Koliński}
\newcommand{\sphinxlogo}{}
\renewcommand{\releasename}{Release}
\makeindex

\makeatletter
\def\PYG@reset{\let\PYG@it=\relax \let\PYG@bf=\relax%
    \let\PYG@ul=\relax \let\PYG@tc=\relax%
    \let\PYG@bc=\relax \let\PYG@ff=\relax}
\def\PYG@tok#1{\csname PYG@tok@#1\endcsname}
\def\PYG@toks#1+{\ifx\relax#1\empty\else%
    \PYG@tok{#1}\expandafter\PYG@toks\fi}
\def\PYG@do#1{\PYG@bc{\PYG@tc{\PYG@ul{%
    \PYG@it{\PYG@bf{\PYG@ff{#1}}}}}}}
\def\PYG#1#2{\PYG@reset\PYG@toks#1+\relax+\PYG@do{#2}}

\expandafter\def\csname PYG@tok@gd\endcsname{\def\PYG@tc##1{\textcolor[rgb]{0.63,0.00,0.00}{##1}}}
\expandafter\def\csname PYG@tok@gu\endcsname{\let\PYG@bf=\textbf\def\PYG@tc##1{\textcolor[rgb]{0.50,0.00,0.50}{##1}}}
\expandafter\def\csname PYG@tok@gt\endcsname{\def\PYG@tc##1{\textcolor[rgb]{0.00,0.25,0.82}{##1}}}
\expandafter\def\csname PYG@tok@gs\endcsname{\let\PYG@bf=\textbf}
\expandafter\def\csname PYG@tok@gr\endcsname{\def\PYG@tc##1{\textcolor[rgb]{1.00,0.00,0.00}{##1}}}
\expandafter\def\csname PYG@tok@cm\endcsname{\let\PYG@it=\textit\def\PYG@tc##1{\textcolor[rgb]{0.25,0.50,0.56}{##1}}}
\expandafter\def\csname PYG@tok@vg\endcsname{\def\PYG@tc##1{\textcolor[rgb]{0.73,0.38,0.84}{##1}}}
\expandafter\def\csname PYG@tok@m\endcsname{\def\PYG@tc##1{\textcolor[rgb]{0.13,0.50,0.31}{##1}}}
\expandafter\def\csname PYG@tok@mh\endcsname{\def\PYG@tc##1{\textcolor[rgb]{0.13,0.50,0.31}{##1}}}
\expandafter\def\csname PYG@tok@cs\endcsname{\def\PYG@tc##1{\textcolor[rgb]{0.25,0.50,0.56}{##1}}\def\PYG@bc##1{\setlength{\fboxsep}{0pt}\colorbox[rgb]{1.00,0.94,0.94}{\strut ##1}}}
\expandafter\def\csname PYG@tok@ge\endcsname{\let\PYG@it=\textit}
\expandafter\def\csname PYG@tok@vc\endcsname{\def\PYG@tc##1{\textcolor[rgb]{0.73,0.38,0.84}{##1}}}
\expandafter\def\csname PYG@tok@il\endcsname{\def\PYG@tc##1{\textcolor[rgb]{0.13,0.50,0.31}{##1}}}
\expandafter\def\csname PYG@tok@go\endcsname{\def\PYG@tc##1{\textcolor[rgb]{0.19,0.19,0.19}{##1}}}
\expandafter\def\csname PYG@tok@cp\endcsname{\def\PYG@tc##1{\textcolor[rgb]{0.00,0.44,0.13}{##1}}}
\expandafter\def\csname PYG@tok@gi\endcsname{\def\PYG@tc##1{\textcolor[rgb]{0.00,0.63,0.00}{##1}}}
\expandafter\def\csname PYG@tok@gh\endcsname{\let\PYG@bf=\textbf\def\PYG@tc##1{\textcolor[rgb]{0.00,0.00,0.50}{##1}}}
\expandafter\def\csname PYG@tok@ni\endcsname{\let\PYG@bf=\textbf\def\PYG@tc##1{\textcolor[rgb]{0.84,0.33,0.22}{##1}}}
\expandafter\def\csname PYG@tok@nl\endcsname{\let\PYG@bf=\textbf\def\PYG@tc##1{\textcolor[rgb]{0.00,0.13,0.44}{##1}}}
\expandafter\def\csname PYG@tok@nn\endcsname{\let\PYG@bf=\textbf\def\PYG@tc##1{\textcolor[rgb]{0.05,0.52,0.71}{##1}}}
\expandafter\def\csname PYG@tok@no\endcsname{\def\PYG@tc##1{\textcolor[rgb]{0.38,0.68,0.84}{##1}}}
\expandafter\def\csname PYG@tok@na\endcsname{\def\PYG@tc##1{\textcolor[rgb]{0.25,0.44,0.63}{##1}}}
\expandafter\def\csname PYG@tok@nb\endcsname{\def\PYG@tc##1{\textcolor[rgb]{0.00,0.44,0.13}{##1}}}
\expandafter\def\csname PYG@tok@nc\endcsname{\let\PYG@bf=\textbf\def\PYG@tc##1{\textcolor[rgb]{0.05,0.52,0.71}{##1}}}
\expandafter\def\csname PYG@tok@nd\endcsname{\let\PYG@bf=\textbf\def\PYG@tc##1{\textcolor[rgb]{0.33,0.33,0.33}{##1}}}
\expandafter\def\csname PYG@tok@ne\endcsname{\def\PYG@tc##1{\textcolor[rgb]{0.00,0.44,0.13}{##1}}}
\expandafter\def\csname PYG@tok@nf\endcsname{\def\PYG@tc##1{\textcolor[rgb]{0.02,0.16,0.49}{##1}}}
\expandafter\def\csname PYG@tok@si\endcsname{\let\PYG@it=\textit\def\PYG@tc##1{\textcolor[rgb]{0.44,0.63,0.82}{##1}}}
\expandafter\def\csname PYG@tok@s2\endcsname{\def\PYG@tc##1{\textcolor[rgb]{0.25,0.44,0.63}{##1}}}
\expandafter\def\csname PYG@tok@vi\endcsname{\def\PYG@tc##1{\textcolor[rgb]{0.73,0.38,0.84}{##1}}}
\expandafter\def\csname PYG@tok@nt\endcsname{\let\PYG@bf=\textbf\def\PYG@tc##1{\textcolor[rgb]{0.02,0.16,0.45}{##1}}}
\expandafter\def\csname PYG@tok@nv\endcsname{\def\PYG@tc##1{\textcolor[rgb]{0.73,0.38,0.84}{##1}}}
\expandafter\def\csname PYG@tok@s1\endcsname{\def\PYG@tc##1{\textcolor[rgb]{0.25,0.44,0.63}{##1}}}
\expandafter\def\csname PYG@tok@gp\endcsname{\let\PYG@bf=\textbf\def\PYG@tc##1{\textcolor[rgb]{0.78,0.36,0.04}{##1}}}
\expandafter\def\csname PYG@tok@sh\endcsname{\def\PYG@tc##1{\textcolor[rgb]{0.25,0.44,0.63}{##1}}}
\expandafter\def\csname PYG@tok@ow\endcsname{\let\PYG@bf=\textbf\def\PYG@tc##1{\textcolor[rgb]{0.00,0.44,0.13}{##1}}}
\expandafter\def\csname PYG@tok@sx\endcsname{\def\PYG@tc##1{\textcolor[rgb]{0.78,0.36,0.04}{##1}}}
\expandafter\def\csname PYG@tok@bp\endcsname{\def\PYG@tc##1{\textcolor[rgb]{0.00,0.44,0.13}{##1}}}
\expandafter\def\csname PYG@tok@c1\endcsname{\let\PYG@it=\textit\def\PYG@tc##1{\textcolor[rgb]{0.25,0.50,0.56}{##1}}}
\expandafter\def\csname PYG@tok@kc\endcsname{\let\PYG@bf=\textbf\def\PYG@tc##1{\textcolor[rgb]{0.00,0.44,0.13}{##1}}}
\expandafter\def\csname PYG@tok@c\endcsname{\let\PYG@it=\textit\def\PYG@tc##1{\textcolor[rgb]{0.25,0.50,0.56}{##1}}}
\expandafter\def\csname PYG@tok@mf\endcsname{\def\PYG@tc##1{\textcolor[rgb]{0.13,0.50,0.31}{##1}}}
\expandafter\def\csname PYG@tok@err\endcsname{\def\PYG@bc##1{\setlength{\fboxsep}{0pt}\fcolorbox[rgb]{1.00,0.00,0.00}{1,1,1}{\strut ##1}}}
\expandafter\def\csname PYG@tok@kd\endcsname{\let\PYG@bf=\textbf\def\PYG@tc##1{\textcolor[rgb]{0.00,0.44,0.13}{##1}}}
\expandafter\def\csname PYG@tok@ss\endcsname{\def\PYG@tc##1{\textcolor[rgb]{0.32,0.47,0.09}{##1}}}
\expandafter\def\csname PYG@tok@sr\endcsname{\def\PYG@tc##1{\textcolor[rgb]{0.14,0.33,0.53}{##1}}}
\expandafter\def\csname PYG@tok@mo\endcsname{\def\PYG@tc##1{\textcolor[rgb]{0.13,0.50,0.31}{##1}}}
\expandafter\def\csname PYG@tok@mi\endcsname{\def\PYG@tc##1{\textcolor[rgb]{0.13,0.50,0.31}{##1}}}
\expandafter\def\csname PYG@tok@kn\endcsname{\let\PYG@bf=\textbf\def\PYG@tc##1{\textcolor[rgb]{0.00,0.44,0.13}{##1}}}
\expandafter\def\csname PYG@tok@o\endcsname{\def\PYG@tc##1{\textcolor[rgb]{0.40,0.40,0.40}{##1}}}
\expandafter\def\csname PYG@tok@kr\endcsname{\let\PYG@bf=\textbf\def\PYG@tc##1{\textcolor[rgb]{0.00,0.44,0.13}{##1}}}
\expandafter\def\csname PYG@tok@s\endcsname{\def\PYG@tc##1{\textcolor[rgb]{0.25,0.44,0.63}{##1}}}
\expandafter\def\csname PYG@tok@kp\endcsname{\def\PYG@tc##1{\textcolor[rgb]{0.00,0.44,0.13}{##1}}}
\expandafter\def\csname PYG@tok@w\endcsname{\def\PYG@tc##1{\textcolor[rgb]{0.73,0.73,0.73}{##1}}}
\expandafter\def\csname PYG@tok@kt\endcsname{\def\PYG@tc##1{\textcolor[rgb]{0.56,0.13,0.00}{##1}}}
\expandafter\def\csname PYG@tok@sc\endcsname{\def\PYG@tc##1{\textcolor[rgb]{0.25,0.44,0.63}{##1}}}
\expandafter\def\csname PYG@tok@sb\endcsname{\def\PYG@tc##1{\textcolor[rgb]{0.25,0.44,0.63}{##1}}}
\expandafter\def\csname PYG@tok@k\endcsname{\let\PYG@bf=\textbf\def\PYG@tc##1{\textcolor[rgb]{0.00,0.44,0.13}{##1}}}
\expandafter\def\csname PYG@tok@se\endcsname{\let\PYG@bf=\textbf\def\PYG@tc##1{\textcolor[rgb]{0.25,0.44,0.63}{##1}}}
\expandafter\def\csname PYG@tok@sd\endcsname{\let\PYG@it=\textit\def\PYG@tc##1{\textcolor[rgb]{0.25,0.44,0.63}{##1}}}

\def\PYGZbs{\char`\\}
\def\PYGZus{\char`\_}
\def\PYGZob{\char`\{}
\def\PYGZcb{\char`\}}
\def\PYGZca{\char`\^}
\def\PYGZam{\char`\&}
\def\PYGZlt{\char`\<}
\def\PYGZgt{\char`\>}
\def\PYGZsh{\char`\#}
\def\PYGZpc{\char`\%}
\def\PYGZdl{\char`\$}
\def\PYGZti{\char`\~}
% for compatibility with earlier versions
\def\PYGZat{@}
\def\PYGZlb{[}
\def\PYGZrb{]}
\makeatother

\begin{document}

\maketitle
\tableofcontents
\phantomsection\label{index::doc}


Contents:


\chapter{Tutorial}
\label{tutorial::doc}\label{tutorial:welcome-to-cabs-s-documentation}\label{tutorial:tutorial}

\section{Calculating heat capacity, $C_v$}
\label{tutorial:calculating-heat-capacity}\begin{gather}
\begin{split}C_v(T) = \frac{\left<E^2\right> - \left<E\right>^2}{T^2}\end{split}\notag\\\begin{split}\end{split}\notag
\end{gather}
\begin{Verbatim}[commandchars=\\\{\}]
\PYG{c}{\PYGZsh{}!/usr/bin/env python}
\PYG{k+kn}{import} \PYG{n+nn}{multiprocessing} \PYG{k+kn}{as} \PYG{n+nn}{mp}
\PYG{k+kn}{import} \PYG{n+nn}{os}
\PYG{k+kn}{import} \PYG{n+nn}{numpy} \PYG{k+kn}{as} \PYG{n+nn}{np}

\PYG{k+kn}{import} \PYG{n+nn}{pycabs}


\PYG{k}{def} \PYG{n+nf}{runCABS}\PYG{p}{(}\PYG{n}{temperature}\PYG{p}{)}\PYG{p}{:}
	\PYG{c}{\PYGZsh{} global for simplify arguments }
	\PYG{k}{global} \PYG{n}{name}\PYG{p}{,} \PYG{n}{sequence}\PYG{p}{,}\PYG{n}{secstr}\PYG{p}{,}\PYG{n}{template} 
	
	\PYG{c}{\PYGZsh{} function for running CABS with different temperatures}
	\PYG{c}{\PYGZsh{} it will compute in directory name+\PYGZus{}+temperature}
	\PYG{n}{here} \PYG{o}{=} \PYG{n}{os}\PYG{o}{.}\PYG{n}{getcwd}\PYG{p}{(}\PYG{p}{)} \PYG{c}{\PYGZsh{} since pycabs changing directories...}
	\PYG{n}{a} \PYG{o}{=} \PYG{n}{pycabs}\PYG{o}{.}\PYG{n}{CABS}\PYG{p}{(}\PYG{n}{sequence}\PYG{p}{,}\PYG{n}{secstr}\PYG{p}{,}\PYG{n}{template}\PYG{p}{,}\PYG{n}{name}\PYG{o}{+}\PYG{l+s}{"}\PYG{l+s}{\PYGZus{}}\PYG{l+s}{"}\PYG{o}{+}\PYG{n+nb}{str}\PYG{p}{(}\PYG{n}{temperature}\PYG{p}{)}\PYG{p}{)}
	\PYG{n}{a}\PYG{o}{.}\PYG{n}{createLatticeReplicas}\PYG{p}{(}\PYG{n}{replicas}\PYG{o}{=}\PYG{l+m+mi}{1}\PYG{p}{)}
	\PYG{n}{a}\PYG{o}{.}\PYG{n}{modeling}\PYG{p}{(}\PYG{n}{Ltemp}\PYG{o}{=}\PYG{n}{temperature}\PYG{p}{,}\PYG{n}{Htemp}\PYG{o}{=}\PYG{n}{temperature}\PYG{p}{,} \PYG{n}{phot}\PYG{o}{=}\PYG{l+m+mi}{85}\PYG{p}{,}\PYG{n}{cycles}\PYG{o}{=}\PYG{l+m+mi}{2}\PYG{p}{)}
	\PYG{c}{\PYGZsh{}remember to come back to {}`here{}` directory}
	\PYG{n}{os}\PYG{o}{.}\PYG{n}{chdir}\PYG{p}{(}\PYG{n}{here}\PYG{p}{)}


\PYG{c}{\PYGZsh{}init these variables \PYGZus{}before\PYGZus{} running cabs}
\PYG{n}{name} \PYG{o}{=} \PYG{l+s}{"}\PYG{l+s}{fnord}\PYG{l+s}{"}
\PYG{c}{\PYGZsh{} we have some template, it has to be as list}
\PYG{n}{template}\PYG{o}{=}\PYG{p}{[}\PYG{l+s}{"}\PYG{l+s}{/home/hydek/pycabs/playground/2pcy.pdb}\PYG{l+s}{"}\PYG{p}{]} 
\PYG{c}{\PYGZsh{} suppose we have porter prediction of sec. str.}
\PYG{n}{sss} \PYG{o}{=}  \PYG{n}{pycabs}\PYG{o}{.}\PYG{n}{parsePorterOutput}\PYG{p}{(}\PYG{l+s}{"}\PYG{l+s}{/home/hydek/pycabs/proba/playground/porter.ss}\PYG{l+s}{"}\PYG{p}{)} 
\PYG{n}{sequence} \PYG{o}{=} \PYG{n}{sss}\PYG{p}{[}\PYG{l+m+mi}{0}\PYG{p}{]}
\PYG{n}{secstr} \PYG{o}{=} \PYG{n}{sss}\PYG{p}{[}\PYG{l+m+mi}{1}\PYG{p}{]}
\PYG{c}{\PYGZsh{} now we have all data required to run CABS}

\PYG{n}{temp\PYGZus{}from} \PYG{o}{=} \PYG{l+m+mf}{2.0}
\PYG{n}{temp\PYGZus{}to}  \PYG{o}{=} \PYG{l+m+mf}{4.0}
\PYG{n}{temp\PYGZus{}interval} \PYG{o}{=} \PYG{l+m+mf}{0.07}
\PYG{n}{temperatures}\PYG{o}{=}\PYG{n}{np}\PYG{o}{.}\PYG{n}{arange}\PYG{p}{(}\PYG{n}{temp\PYGZus{}from}\PYG{p}{,}\PYG{n}{temp\PYGZus{}to}\PYG{p}{,}\PYG{n}{temp\PYGZus{}interval}\PYG{p}{)} \PYG{c}{\PYGZsh{} ranges of temperature}

\PYG{c}{\PYGZsh{} create thread pool  with two parallel threads}
\PYG{n}{pool} \PYG{o}{=} \PYG{n}{mp}\PYG{o}{.}\PYG{n}{Pool}\PYG{p}{(}\PYG{n}{processes}\PYG{o}{=}\PYG{l+m+mi}{2}\PYG{p}{)}
\PYG{n}{pool}\PYG{o}{.}\PYG{n}{map}\PYG{p}{(}\PYG{n}{runCABS}\PYG{p}{,}\PYG{n}{temperatures}\PYG{p}{)} \PYG{c}{\PYGZsh{} run cabs threads}

\PYG{c}{\PYGZsh{} HERE IS THE END OF PART WHERE WE RUN CABS in parallel fashion. }

\PYG{c}{\PYGZsh{} Now you can do something with output data, we'll calculate heat capacity, Cv:}
\PYG{n}{cv} \PYG{o}{=} \PYG{n}{np}\PYG{o}{.}\PYG{n}{empty}\PYG{p}{(}\PYG{n+nb}{len}\PYG{p}{(}\PYG{n}{temperatures}\PYG{p}{)}\PYG{p}{)}
\PYG{k}{for} \PYG{n}{i} \PYG{o+ow}{in} \PYG{n+nb}{range}\PYG{p}{(}\PYG{n+nb}{len}\PYG{p}{(}\PYG{n}{temperatures}\PYG{p}{)}\PYG{p}{)}\PYG{p}{:}
	\PYG{n}{t} \PYG{o}{=} \PYG{n}{temperatures}\PYG{p}{[}\PYG{n}{i}\PYG{p}{]}
	\PYG{n}{e\PYGZus{}path} \PYG{o}{=} \PYG{n}{os}\PYG{o}{.}\PYG{n}{path}\PYG{o}{.}\PYG{n}{join}\PYG{p}{(}\PYG{n}{name}\PYG{o}{+}\PYG{l+s}{'}\PYG{l+s}{\PYGZus{}}\PYG{l+s}{'}\PYG{o}{+}\PYG{n+nb}{str}\PYG{p}{(}\PYG{n}{t}\PYG{p}{)}\PYG{p}{,}\PYG{l+s}{'}\PYG{l+s}{ENERGY}\PYG{l+s}{'}\PYG{p}{)}
	\PYG{n}{energy} \PYG{o}{=} \PYG{n}{np}\PYG{o}{.}\PYG{n}{fromfile}\PYG{p}{(}\PYG{n}{e\PYGZus{}path}\PYG{p}{,}\PYG{n}{sep}\PYG{o}{=}\PYG{l+s}{'}\PYG{l+s+se}{\PYGZbs{}n}\PYG{l+s}{'}\PYG{p}{)} \PYG{c}{\PYGZsh{} read ENERGY data into array {}`energy{}`}
	\PYG{n}{avg\PYGZus{}energy2} \PYG{o}{=} \PYG{n}{np}\PYG{o}{.}\PYG{n}{average}\PYG{p}{(}\PYG{n}{energy}\PYG{o}{*}\PYG{n}{energy}\PYG{p}{)} \PYG{c}{\PYGZsh{} \PYGZlt{}E\PYGZca{}2\PYGZgt{}}
	\PYG{n}{avg\PYGZus{}energy} \PYG{o}{=} \PYG{n}{np}\PYG{o}{.}\PYG{n}{average}\PYG{p}{(}\PYG{n}{energy}\PYG{p}{)}		    \PYG{c}{\PYGZsh{} \PYGZlt{}E\PYGZgt{}\PYGZca{}2}
	\PYG{n}{cv}\PYG{p}{[}\PYG{n}{i}\PYG{p}{]} \PYG{o}{=} \PYG{p}{(}\PYG{n}{avg\PYGZus{}energy2} \PYG{o}{-} \PYG{n}{avg\PYGZus{}energy}\PYG{o}{*}\PYG{n}{avg\PYGZus{}energy}\PYG{p}{)}\PYG{o}{/}\PYG{p}{(}\PYG{n}{t}\PYG{o}{*}\PYG{n}{t}\PYG{p}{)} \PYG{c}{\PYGZsh{} (\PYGZlt{}E\PYGZca{}2\PYGZgt{} - \PYGZlt{}E\PYGZgt{}\PYGZca{}2) / T\PYGZca{}2}
\PYG{c}{\PYGZsh{} now we have heat capacity in cv array	}

\PYG{c}{\PYGZsh{} ... and display plot}
\PYG{k+kn}{from} \PYG{n+nn}{pylab} \PYG{k+kn}{import} \PYG{o}{*}
\PYG{n}{xlabel}\PYG{p}{(}\PYG{l+s}{r'}\PYG{l+s}{temperature \PYGZdl{}T\PYGZdl{}}\PYG{l+s}{'}\PYG{p}{)}
\PYG{n}{ylabel}\PYG{p}{(}\PYG{l+s}{r'}\PYG{l+s}{heat capacity \PYGZdl{}C\PYGZus{}v = (}\PYG{l+s}{\PYGZbs{}}\PYG{l+s}{left\PYGZlt{}E\PYGZca{}2}\PYG{l+s}{\PYGZbs{}}\PYG{l+s}{right\PYGZgt{} - }\PYG{l+s}{\PYGZbs{}}\PYG{l+s}{left\PYGZlt{}E}\PYG{l+s}{\PYGZbs{}}\PYG{l+s}{right\PYGZgt{}\PYGZca{}2)/T\PYGZca{}2\PYGZdl{}}\PYG{l+s}{'} \PYG{p}{)}
\PYG{n}{xlim}\PYG{p}{(}\PYG{n}{temp\PYGZus{}from}\PYG{p}{,}\PYG{n}{temp\PYGZus{}to}\PYG{p}{)} \PYG{c}{\PYGZsh{} xrange}
\PYG{n}{plot}\PYG{p}{(}\PYG{n}{temperatures}\PYG{p}{,}\PYG{n}{cv}\PYG{p}{)}
\PYG{n}{show}\PYG{p}{(}\PYG{p}{)}

\PYG{c}{\PYGZsh{}remember that you have name+\PYGZus{}+temperature directories, delete it or sth}
\end{Verbatim}


\section{Monitoring of CABS energy during simulation}
\label{tutorial:monitoring-of-cabs-energy-during-simulation}
\begin{Verbatim}[commandchars=\\\{\}]
\PYG{c}{\PYGZsh{}!/usr/bin/env python}
\PYG{k+kn}{from} \PYG{n+nn}{pylab} \PYG{k+kn}{import} \PYG{o}{*}
\PYG{k+kn}{from} \PYG{n+nn}{sys} \PYG{k+kn}{import} \PYG{n}{argv}
\PYG{k+kn}{import} \PYG{n+nn}{os}
\PYG{k+kn}{import} \PYG{n+nn}{time}
\PYG{k+kn}{import} \PYG{n+nn}{numpy} \PYG{k+kn}{as} \PYG{n+nn}{np}
\PYG{k+kn}{import} \PYG{n+nn}{pycabs}

\PYG{k}{class} \PYG{n+nc}{Energy}\PYG{p}{(}\PYG{n}{pycabs}\PYG{o}{.}\PYG{n}{Calculate}\PYG{p}{)}\PYG{p}{:}
    \PYG{k}{def} \PYG{n+nf}{calculate}\PYG{p}{(}\PYG{n+nb+bp}{self}\PYG{p}{,}\PYG{n}{data}\PYG{p}{)}\PYG{p}{:}
        \PYG{k}{for} \PYG{n}{i} \PYG{o+ow}{in} \PYG{n}{data}\PYG{p}{:}
            \PYG{n+nb+bp}{self}\PYG{o}{.}\PYG{n}{out}\PYG{o}{.}\PYG{n}{append}\PYG{p}{(}\PYG{n+nb}{float}\PYG{p}{(}\PYG{n}{i}\PYG{p}{)}\PYG{p}{)} \PYG{c}{\PYGZsh{} ENERGY file contains one value in a row}
            
\PYG{n}{out} \PYG{o}{=} \PYG{p}{[}\PYG{p}{]}						
\PYG{n}{calc} \PYG{o}{=} \PYG{n}{Energy}\PYG{p}{(}\PYG{n}{out}\PYG{p}{)} \PYG{c}{\PYGZsh{} out is dynamically updated }
\PYG{n}{m}\PYG{o}{=}\PYG{n}{pycabs}\PYG{o}{.}\PYG{n}{Monitor}\PYG{p}{(}\PYG{n}{os}\PYG{o}{.}\PYG{n}{path}\PYG{o}{.}\PYG{n}{join}\PYG{p}{(}\PYG{n}{argv}\PYG{p}{[}\PYG{l+m+mi}{1}\PYG{p}{]}\PYG{p}{,}\PYG{l+s}{"}\PYG{l+s}{ENERGY}\PYG{l+s}{"}\PYG{p}{)}\PYG{p}{,}\PYG{n}{calc}\PYG{p}{)}
\PYG{n}{m}\PYG{o}{.}\PYG{n}{daemon} \PYG{o}{=} \PYG{n+nb+bp}{True}
\PYG{n}{m}\PYG{o}{.}\PYG{n}{start}\PYG{p}{(}\PYG{p}{)}



\PYG{n}{ion}\PYG{p}{(}\PYG{p}{)}
\PYG{n}{y} \PYG{o}{=} \PYG{n}{zeros}\PYG{p}{(}\PYG{l+m+mi}{1}\PYG{p}{)}
\PYG{n}{x} \PYG{o}{=} \PYG{n}{zeros}\PYG{p}{(}\PYG{l+m+mi}{1}\PYG{p}{)}
\PYG{n}{line}\PYG{p}{,} \PYG{o}{=} \PYG{n}{plot}\PYG{p}{(}\PYG{n}{x}\PYG{p}{,}\PYG{n}{y}\PYG{p}{)}
\PYG{n}{xlabel}\PYG{p}{(}\PYG{l+s}{'}\PYG{l+s}{CABS time step}\PYG{l+s}{'}\PYG{p}{)}
\PYG{n}{ylabel}\PYG{p}{(}\PYG{l+s}{'}\PYG{l+s}{CABS energy}\PYG{l+s}{'}\PYG{p}{)}

\PYG{k}{while} \PYG{l+m+mi}{1}\PYG{p}{:}
    \PYG{n}{time}\PYG{o}{.}\PYG{n}{sleep}\PYG{p}{(}\PYG{l+m+mi}{1}\PYG{p}{)}
    \PYG{n}{y} \PYG{o}{=} \PYG{n}{np}\PYG{o}{.}\PYG{n}{asarray}\PYG{p}{(}\PYG{n}{out}\PYG{p}{)}
    \PYG{n}{x} \PYG{o}{=} \PYG{n+nb}{xrange}\PYG{p}{(}\PYG{l+m+mi}{0}\PYG{p}{,}\PYG{n+nb}{len}\PYG{p}{(}\PYG{n}{out}\PYG{p}{)}\PYG{p}{)}
    \PYG{n}{axis}\PYG{p}{(}\PYG{p}{[}\PYG{l+m+mi}{0}\PYG{p}{,} \PYG{n}{amax}\PYG{p}{(}\PYG{n}{x}\PYG{p}{)}\PYG{o}{+}\PYG{l+m+mi}{1}\PYG{p}{,} \PYG{n}{amin}\PYG{p}{(}\PYG{n}{y}\PYG{p}{)}\PYG{o}{-}\PYG{l+m+mi}{5}\PYG{p}{,} \PYG{n}{amax}\PYG{p}{(}\PYG{n}{y}\PYG{p}{)}\PYG{o}{+}\PYG{l+m+mi}{5} \PYG{p}{]}\PYG{p}{)}
    \PYG{n}{line}\PYG{o}{.}\PYG{n}{set\PYGZus{}ydata}\PYG{p}{(}\PYG{n}{y}\PYG{p}{)}  \PYG{c}{\PYGZsh{} update the data}
    \PYG{n}{line}\PYG{o}{.}\PYG{n}{set\PYGZus{}xdata}\PYG{p}{(}\PYG{n}{x}\PYG{p}{)}
    \PYG{n}{draw}\PYG{p}{(}\PYG{p}{)}
\end{Verbatim}


\section{Monitoring of end-to-end distance of chain during simulation}
\label{tutorial:monitoring-of-end-to-end-distance-of-chain-during-simulation}
\begin{Verbatim}[commandchars=\\\{\}]
\PYG{c}{\PYGZsh{}!/usr/bin/env python}
\PYG{k+kn}{from} \PYG{n+nn}{pylab} \PYG{k+kn}{import} \PYG{o}{*}
\PYG{k+kn}{from} \PYG{n+nn}{sys} \PYG{k+kn}{import} \PYG{n}{argv}
\PYG{k+kn}{import} \PYG{n+nn}{time}
\PYG{k+kn}{import} \PYG{n+nn}{os}
\PYG{k+kn}{import} \PYG{n+nn}{numpy} \PYG{k+kn}{as} \PYG{n+nn}{np}
\PYG{k+kn}{import} \PYG{n+nn}{pycabs}

\PYG{k}{class} \PYG{n+nc}{E2E}\PYG{p}{(}\PYG{n}{pycabs}\PYG{o}{.}\PYG{n}{Calculate}\PYG{p}{)}\PYG{p}{:}
    \PYG{k}{def} \PYG{n+nf}{calculate}\PYG{p}{(}\PYG{n+nb+bp}{self}\PYG{p}{,}\PYG{n}{data}\PYG{p}{)}\PYG{p}{:}
        \PYG{n}{models} \PYG{o}{=} \PYG{n+nb+bp}{self}\PYG{o}{.}\PYG{n}{processTrajectory}\PYG{p}{(}\PYG{n}{data}\PYG{p}{)}
        \PYG{k}{for} \PYG{n}{m} \PYG{o+ow}{in} \PYG{n}{models}\PYG{p}{:}
            \PYG{n}{first} \PYG{o}{=} \PYG{n}{m}\PYG{p}{[}\PYG{l+m+mi}{0}\PYG{p}{:}\PYG{l+m+mi}{3}\PYG{p}{]}
            \PYG{n}{last} \PYG{o}{=} \PYG{n}{m}\PYG{p}{[}\PYG{o}{-}\PYG{l+m+mi}{3}\PYG{p}{:}\PYG{p}{]}

            \PYG{n}{x} \PYG{o}{=} \PYG{n}{first}\PYG{p}{[}\PYG{l+m+mi}{0}\PYG{p}{]}\PYG{o}{-}\PYG{n}{last}\PYG{p}{[}\PYG{l+m+mi}{0}\PYG{p}{]}
            \PYG{n}{y} \PYG{o}{=} \PYG{n}{first}\PYG{p}{[}\PYG{l+m+mi}{1}\PYG{p}{]}\PYG{o}{-}\PYG{n}{last}\PYG{p}{[}\PYG{l+m+mi}{1}\PYG{p}{]}
            \PYG{n}{z} \PYG{o}{=} \PYG{n}{first}\PYG{p}{[}\PYG{l+m+mi}{2}\PYG{p}{]}\PYG{o}{-}\PYG{n}{last}\PYG{p}{[}\PYG{l+m+mi}{2}\PYG{p}{]}
            \PYG{n+nb+bp}{self}\PYG{o}{.}\PYG{n}{out}\PYG{o}{.}\PYG{n}{append}\PYG{p}{(}\PYG{n}{x}\PYG{o}{*}\PYG{n}{x}\PYG{o}{+}\PYG{n}{y}\PYG{o}{*}\PYG{n}{y}\PYG{o}{+}\PYG{n}{z}\PYG{o}{*}\PYG{n}{z}\PYG{p}{)}            
            
\PYG{n}{out} \PYG{o}{=} \PYG{p}{[}\PYG{p}{]}						
\PYG{n}{calc} \PYG{o}{=} \PYG{n}{E2E}\PYG{p}{(}\PYG{n}{out}\PYG{p}{)} \PYG{c}{\PYGZsh{} out is dynamically updated }
\PYG{n}{m}\PYG{o}{=}\PYG{n}{pycabs}\PYG{o}{.}\PYG{n}{Monitor}\PYG{p}{(}\PYG{n}{os}\PYG{o}{.}\PYG{n}{path}\PYG{o}{.}\PYG{n}{join}\PYG{p}{(}\PYG{n}{argv}\PYG{p}{[}\PYG{l+m+mi}{1}\PYG{p}{]}\PYG{p}{,}\PYG{l+s}{"}\PYG{l+s}{TRAF}\PYG{l+s}{"}\PYG{p}{)}\PYG{p}{,}\PYG{n}{calc}\PYG{p}{)}
\PYG{n}{m}\PYG{o}{.}\PYG{n}{daemon} \PYG{o}{=} \PYG{n+nb+bp}{True}
\PYG{n}{m}\PYG{o}{.}\PYG{n}{start}\PYG{p}{(}\PYG{p}{)}


\PYG{n}{ion}\PYG{p}{(}\PYG{p}{)}
\PYG{n}{y} \PYG{o}{=} \PYG{n}{zeros}\PYG{p}{(}\PYG{l+m+mi}{1}\PYG{p}{)}
\PYG{n}{x} \PYG{o}{=} \PYG{n}{zeros}\PYG{p}{(}\PYG{l+m+mi}{1}\PYG{p}{)}
\PYG{n}{line}\PYG{p}{,} \PYG{o}{=} \PYG{n}{plot}\PYG{p}{(}\PYG{n}{x}\PYG{p}{,}\PYG{n}{y}\PYG{p}{)}
\PYG{n}{xlabel}\PYG{p}{(}\PYG{l+s}{'}\PYG{l+s}{CABS time step}\PYG{l+s}{'}\PYG{p}{)}
\PYG{n}{ylabel}\PYG{p}{(}\PYG{l+s}{'}\PYG{l+s}{square of end to end distance}\PYG{l+s}{'}\PYG{p}{)}

\PYG{k}{while} \PYG{l+m+mi}{1}\PYG{p}{:}
    \PYG{n}{time}\PYG{o}{.}\PYG{n}{sleep}\PYG{p}{(}\PYG{l+m+mi}{1}\PYG{p}{)}
    \PYG{n}{y} \PYG{o}{=} \PYG{n}{np}\PYG{o}{.}\PYG{n}{asarray}\PYG{p}{(}\PYG{n}{out}\PYG{p}{)}
    \PYG{n}{x} \PYG{o}{=} \PYG{n+nb}{xrange}\PYG{p}{(}\PYG{l+m+mi}{0}\PYG{p}{,}\PYG{n+nb}{len}\PYG{p}{(}\PYG{n}{out}\PYG{p}{)}\PYG{p}{)}
    \PYG{n}{axis}\PYG{p}{(}\PYG{p}{[}\PYG{l+m+mi}{0}\PYG{p}{,} \PYG{n}{amax}\PYG{p}{(}\PYG{n}{x}\PYG{p}{)}\PYG{o}{+}\PYG{l+m+mi}{1}\PYG{p}{,} \PYG{n}{amin}\PYG{p}{(}\PYG{n}{y}\PYG{p}{)}\PYG{o}{-}\PYG{l+m+mi}{5}\PYG{p}{,} \PYG{n}{amax}\PYG{p}{(}\PYG{n}{y}\PYG{p}{)}\PYG{o}{+}\PYG{l+m+mi}{5} \PYG{p}{]}\PYG{p}{)}
    \PYG{n}{line}\PYG{o}{.}\PYG{n}{set\PYGZus{}ydata}\PYG{p}{(}\PYG{n}{y}\PYG{p}{)}  \PYG{c}{\PYGZsh{} update the data}
    \PYG{n}{line}\PYG{o}{.}\PYG{n}{set\PYGZus{}xdata}\PYG{p}{(}\PYG{n}{x}\PYG{p}{)}
    \PYG{n}{draw}\PYG{p}{(}\PYG{p}{)}
\end{Verbatim}


\chapter{pyCABS API}
\label{api:module-pycabs}\label{api::doc}\label{api:pycabs-api}\index{pycabs (module)}
pyCABS Copyright (C) 2012 Michal Jamroz \textless{}\href{mailto:jamroz@chem.uw.edu.pl}{jamroz@chem.uw.edu.pl}\textgreater{}

This program is free software: you can redistribute it and/or modify
it under the terms of the GNU General Public License as published by
the Free Software Foundation, either version 3 of the License, or
(at your option) any later version.

This program is distributed in the hope that it will be useful,
but WITHOUT ANY WARRANTY; without even the implied warranty of
MERCHANTABILITY or FITNESS FOR A PARTICULAR PURPOSE.  See the
GNU General Public License for more details.

You should have received a copy of the GNU General Public License
along with this program.  If not, see \textless{}\href{http://www.gnu.org/licenses/}{http://www.gnu.org/licenses/}\textgreater{}.
\index{CABS (class in pycabs)}

\begin{fulllineitems}
\phantomsection\label{api:pycabs.CABS}\pysiglinewithargsret{\strong{class }\code{pycabs.}\bfcode{CABS}}{\emph{sequence}, \emph{secondary\_structure}, \emph{templates\_filenames}, \emph{project\_name}}{}
CABS main class.
\begin{quote}\begin{description}
\item[{Parameters}] \leavevmode\begin{itemize}
\item {} 
\textbf{sequence} (\emph{string}) -- one line sequence of the target protein

\item {} 
\textbf{secondary\_structure} (\emph{string}) -- one line secondary structure for the target protein

\item {} 
\textbf{templates\_filenames} (\emph{list}) -- path to 3D protein model templates in pdb file format which you want to use for modeling. C\(\alpha\) numbering in templates must be aligned to target sequence

\item {} 
\textbf{project\_name} (\emph{string}) -- project\_name and working directory name (uniq)

\end{itemize}

\end{description}\end{quote}
\index{calcConstraints() (pycabs.CABS method)}

\begin{fulllineitems}
\phantomsection\label{api:pycabs.CABS.calcConstraints}\pysiglinewithargsret{\bfcode{calcConstraints}}{\emph{exclude\_residues=}\optional{}, \emph{other\_constraints=}\optional{}}{}
Calculate distance constraints using templates 3D models.
\begin{quote}\begin{description}
\item[{Parameters}] \leavevmode\begin{itemize}
\item {} 
\textbf{exclude\_residues} (\emph{list}) -- indexes of residues without constrains

\item {} 
\textbf{constrains} (\emph{other}) -- user-defined constrains as list of tuples: (residue\_i\_index,residue\_j\_index,constraint\_strength)

\end{itemize}

\end{description}\end{quote}

\end{fulllineitems}

\index{convertPdbToDcd() (pycabs.CABS method)}

\begin{fulllineitems}
\phantomsection\label{api:pycabs.CABS.convertPdbToDcd}\pysiglinewithargsret{\bfcode{convertPdbToDcd}}{\emph{catdcd\_path='/home/hydek/pycabs/FF/catdcd'}}{}
This is only simple wrapper to CatDCD software (\href{http://www.ks.uiuc.edu/Development/MDTools/catdcd/}{http://www.ks.uiuc.edu/Development/MDTools/catdcd/}), 
could be usable since *.dcd binary format is few times lighter than pdb, and many python libraries 
(ProDy, MDAnalysis) use *.dcd as trajectory input format.
Before use, download CatDCD from \href{http://www.ks.uiuc.edu/Development/MDTools/catdcd/}{http://www.ks.uiuc.edu/Development/MDTools/catdcd/} and modify catdcd\_path.

\end{fulllineitems}

\index{createLatticeReplicas() (pycabs.CABS method)}

\begin{fulllineitems}
\phantomsection\label{api:pycabs.CABS.createLatticeReplicas}\pysiglinewithargsret{\bfcode{createLatticeReplicas}}{\emph{start\_structures\_fn=}\optional{}, \emph{replicas=20}}{}
Create protein models projected onto CABS lattice, which will be used as replicas.
\begin{quote}\begin{description}
\item[{Parameters}] \leavevmode\begin{itemize}
\item {} 
\textbf{start\_structures\_fn} (\emph{list}) -- list of paths to pdb files which should be used instead of templates models.  This parameter is optional, and probably not often used. Without it script creates replicas from templates files.

\item {} 
\textbf{replicas} (\emph{integer}) -- define number of replicas in CABS simulation. However 20 is optimal for most cases, and you don't need to change it in protein modeling case.

\end{itemize}

\end{description}\end{quote}

\begin{notice}{note}{Note:}
If number of replicas is smaller than number of templates - program will create replicas using first \emph{replicas} templates. If there is less templates than replicas, they are creating sequentially using template models.
\end{notice}

\end{fulllineitems}

\index{getEnergy() (pycabs.CABS method)}

\begin{fulllineitems}
\phantomsection\label{api:pycabs.CABS.getEnergy}\pysiglinewithargsret{\bfcode{getEnergy}}{}{}
Read CABS energy values into list
\begin{quote}\begin{description}
\item[{Returns}] \leavevmode
list of models energy

\end{description}\end{quote}

\end{fulllineitems}

\index{getTraCoordinates() (pycabs.CABS method)}

\begin{fulllineitems}
\phantomsection\label{api:pycabs.CABS.getTraCoordinates}\pysiglinewithargsret{\bfcode{getTraCoordinates}}{}{}
Read trajectory file into 2D list of coordinates
\begin{quote}\begin{description}
\item[{Returns}] \leavevmode
2D list of trajectory coordinates

\end{description}\end{quote}

\end{fulllineitems}

\index{rng\_seed (pycabs.CABS attribute)}

\begin{fulllineitems}
\phantomsection\label{api:pycabs.CABS.rng_seed}\pysigline{\bfcode{rng\_seed}\strong{ = None}}
seed for random generator

\end{fulllineitems}

\index{trafToPdb() (pycabs.CABS method)}

\begin{fulllineitems}
\phantomsection\label{api:pycabs.CABS.trafToPdb}\pysiglinewithargsret{\bfcode{trafToPdb}}{\emph{output\_filename='TRAF.pdb'}}{}
Convert TRAF CABS pseudotrajectory file format into multimodel pdb

\end{fulllineitems}


\end{fulllineitems}

\index{Calculate (class in pycabs)}

\begin{fulllineitems}
\phantomsection\label{api:pycabs.Calculate}\pysiglinewithargsret{\strong{class }\code{pycabs.}\bfcode{Calculate}}{\emph{output}}{}
Inherit if you want to process data used with {\hyperref[api:pycabs.Monitor]{\code{Monitor}}} class.
\begin{quote}\begin{description}
\item[{Parameters}] \leavevmode
\textbf{output} (\emph{array/list}) -- output array with calculated results

\end{description}\end{quote}
\index{processTrajectory() (pycabs.Calculate method)}

\begin{fulllineitems}
\phantomsection\label{api:pycabs.Calculate.processTrajectory}\pysiglinewithargsret{\bfcode{processTrajectory}}{\emph{data}}{}
Use it in \emph{calculate} method if you parsing TRAF file, and want to calculate something on structure
\begin{quote}\begin{description}
\item[{Returns}] \leavevmode
array of model coordinates

\end{description}\end{quote}

\end{fulllineitems}


\end{fulllineitems}

\index{Errors}

\begin{fulllineitems}
\phantomsection\label{api:pycabs.Errors}\pysiglinewithargsret{\strong{exception }\code{pycabs.}\bfcode{Errors}}{\emph{value}}{}
Simple error messages

\end{fulllineitems}

\index{Info (class in pycabs)}

\begin{fulllineitems}
\phantomsection\label{api:pycabs.Info}\pysiglinewithargsret{\strong{class }\code{pycabs.}\bfcode{Info}}{\emph{text}}{}
Simple message system

\end{fulllineitems}

\index{Monitor (class in pycabs)}

\begin{fulllineitems}
\phantomsection\label{api:pycabs.Monitor}\pysiglinewithargsret{\strong{class }\code{pycabs.}\bfcode{Monitor}}{\emph{filename}, \emph{calculate}}{}
Class for monitoring of CABS output data. You can run it and dynamically update output arrays with calculated results.
\begin{quote}\begin{description}
\item[{Parameters}] \leavevmode
\textbf{calculate} ({\hyperref[api:pycabs.Calculate]{\code{Calculate}}}) -- what to do with gathered data ?

\end{description}\end{quote}
\index{daemon (pycabs.Monitor attribute)}

\begin{fulllineitems}
\phantomsection\label{api:pycabs.Monitor.daemon}\pysigline{\bfcode{daemon}\strong{ = None}}
if True, it will terminate when script terminates

\end{fulllineitems}

\index{run() (pycabs.Monitor method)}

\begin{fulllineitems}
\phantomsection\label{api:pycabs.Monitor.run}\pysiglinewithargsret{\bfcode{run}}{}{}
Run monitor in background

\end{fulllineitems}

\index{terminate() (pycabs.Monitor method)}

\begin{fulllineitems}
\phantomsection\label{api:pycabs.Monitor.terminate}\pysiglinewithargsret{\bfcode{terminate}}{}{}
Terminate monitor

\end{fulllineitems}


\end{fulllineitems}

\index{Template (class in pycabs)}

\begin{fulllineitems}
\phantomsection\label{api:pycabs.Template}\pysiglinewithargsret{\strong{class }\code{pycabs.}\bfcode{Template}}{\emph{filename}}{}
Class used for template storage of atom positions and distance calculation
\begin{quote}\begin{description}
\item[{Parameters}] \leavevmode
\textbf{filename} -- path to file with template (in PDB format)

\end{description}\end{quote}
\index{distance() (pycabs.Template method)}

\begin{fulllineitems}
\phantomsection\label{api:pycabs.Template.distance}\pysiglinewithargsret{\bfcode{distance}}{\emph{idx\_i}, \emph{idx\_j}}{}~\begin{quote}\begin{description}
\item[{Parameters}] \leavevmode\begin{itemize}
\item {} 
\textbf{idx\_i} (\emph{integer}) -- residue index (as in target sequence numbering)

\item {} 
\textbf{idx\_j} (\emph{integer}) -- residue index (as in target sequence numbering)

\end{itemize}

\item[{Returns}] \leavevmode
euclidean distance between C\(\alpha\)(i) and C\(\alpha\)(j)

\end{description}\end{quote}

\end{fulllineitems}


\end{fulllineitems}

\index{parsePorterOutput() (in module pycabs)}

\begin{fulllineitems}
\phantomsection\label{api:pycabs.parsePorterOutput}\pysiglinewithargsret{\code{pycabs.}\bfcode{parsePorterOutput}}{\emph{porter\_output\_fn}}{}
Porter (protein secondary stucture prediction, \href{http://distill.ucd.ie/porter/}{http://distill.ucd.ie/porter/}) output parser. 
Porter emailed output looks like:

\begin{Verbatim}[commandchars=\\\{\}]
\PYG{n}{IDVLLGADDGSLAFVPSEFSISPGEKIVFKNNAGFPHNIVFDEDSIPSGVDASKISMSEE}
\PYG{n}{CEEEECCCCCCCCEECCEEEECCCCEEEEEECCCCCEEEEECCCCCCCCCCHHHHCCCCC}



\PYG{n}{DLLNAKGETFEVALSNKGEYSFYCSPHQGAGMVGKVTVN}
\PYG{n}{CCECCCCCEEEEECCCCEEEEEECCHHHHCCCEEEEEEC}
\end{Verbatim}
\begin{quote}\begin{description}
\item[{Parameters}] \leavevmode
\textbf{porter\_output\_fn} (\emph{string}) -- path to the porter output file

\item[{Returns}] \leavevmode
tuple (sequence, secondary\_structure)

\end{description}\end{quote}

\end{fulllineitems}

\index{parsePsipredOutput() (in module pycabs)}

\begin{fulllineitems}
\phantomsection\label{api:pycabs.parsePsipredOutput}\pysiglinewithargsret{\code{pycabs.}\bfcode{parsePsipredOutput}}{\emph{psipred\_output\_fn}}{}
Psipred (protein secondary structure prediction, \href{http://bioinf.cs.ucl.ac.uk/psipred/}{http://bioinf.cs.ucl.ac.uk/psipred/}) output parser. 
Psipred output looks like:

\begin{Verbatim}[commandchars=\\\{\}]
\textgreater{} head psipred.ss
1 P C   1.000  0.000  0.000
2 K C   0.665  0.000  0.459
3 A E   0.018  0.000  0.991
4 L E   0.008  0.000  0.997
5 I E   0.002  0.000  0.998
6 V E   0.003  0.000  0.999
7 Y E   0.033  0.000  0.981
\end{Verbatim}
\begin{quote}\begin{description}
\item[{Parameters}] \leavevmode
\textbf{psipred\_output\_fn} (\emph{string}) -- path to the psipred output file

\item[{Returns}] \leavevmode
tuple (sequence, secondary\_structure)

\end{description}\end{quote}

\end{fulllineitems}

\index{rmsd() (in module pycabs)}

\begin{fulllineitems}
\phantomsection\label{api:pycabs.rmsd}\pysiglinewithargsret{\code{pycabs.}\bfcode{rmsd}}{\emph{reference}, \emph{arr}}{}
Calculate coordinate Root Mean Square Deviation between two sets of coordinates.
\begin{gather}
\begin{split}cRMSD = \sqrt{ \sum_{i=1}^N \|x_{i} - y_{i}\|^2 \over N}\end{split}\notag
\end{gather}\begin{quote}\begin{description}
\item[{Parameters}] \leavevmode\begin{itemize}
\item {} 
\textbf{reference} (\emph{list}) -- 1D list of coordinates (length of 3N)

\item {} 
\textbf{arr} (\emph{list}) -- 1D list of coordinates (length of 3N)

\end{itemize}

\item[{Returns}] \leavevmode
RMSD value after optimal superimposition of two structures

\end{description}\end{quote}

\end{fulllineitems}



\chapter{Indices and tables}
\label{index:indices-and-tables}\begin{itemize}
\item {} 
\emph{genindex}

\item {} 
\emph{modindex}

\item {} 
\emph{search}

\end{itemize}


\renewcommand{\indexname}{Python Module Index}
\begin{theindex}
\def\bigletter#1{{\Large\sffamily#1}\nopagebreak\vspace{1mm}}
\bigletter{p}
\item {\texttt{pycabs}}, \pageref{api:module-pycabs}
\end{theindex}

\renewcommand{\indexname}{Index}
\printindex
\end{document}
